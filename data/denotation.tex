%%================================================
%% Filename: denotation.tex
%% Encoding: UTF-8
%% Author: Yuan Xiaoshuai - yxshuai@gmail.com
%% Created: 2012-01-14 13:44
%% Last modified: 2019-11-05 15:46
%%================================================
\begin{denotation}

\item[HPC] 高性能计算 (High Performance Computing)
\item[cluster] 集群
\item[Itanium] 安腾
\item[SMP] 对称多处理
\item[API] 应用程序编程接口
\item[PI]	聚酰亚胺
\item[MPI]	聚酰亚胺模型化合物,N-苯基邻苯酰亚胺
\item[PBI]	聚苯并咪唑
\item[MPBI]	聚苯并咪唑模型化合物,N-苯基苯并咪唑
\item[PY]	聚吡咙
\item[PMDA-BDA]	均苯四酸二酐与联苯四胺合成的聚吡咙薄膜
\item[$\Delta G$]  	活化自由能~(Activation Free Energy)
\item [$\chi$] 传输系数~(Transmission Coefficient)
\item[$E$] 能量
\item[$m$] 质量
\item[$c$] 光速
\item[$P$] 概率
\item[$T$] 时间
\item[$v$] 速度
\item[劝学] 君子曰:学不可以已。青,取之于蓝,而青于蓝;冰,水为之,而寒于水。
木直中绳。(车柔)以为轮,其曲中规。虽有槁暴,不复挺者,(车柔)使之然也。故木
受绳则直, 金就砺则利,君子博学而日参省乎己,则知明而行无过矣。吾尝终日而思矣
,  不如须臾之所学也;吾尝(足齐)而望矣,不如登高之博见也。登高而招,臂非加长
也,  而见者远;  顺风而呼,  声非加疾也,而闻者彰。假舆马者,非利足也,而致千
里;假舟楫者,非能水也,而绝江河,  君子生非异也,善假于物也。积土成山,风雨兴
焉;积水成渊,蛟龙生焉;积善成德,而神明自得,圣心备焉。故不积跬步,无以至千里
;不积小流,无以成江海。骐骥一跃,不能十步;驽马十驾,功在不舍。锲而舍之,朽木
不折;  锲而不舍,金石可镂。蚓无爪牙之利,筋骨之强,上食埃土,下饮黄泉,用心一
也。蟹六跪而二螯,非蛇鳝之穴无可寄托者,用心躁也。——荀况
\end{denotation}
