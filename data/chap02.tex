%%================================================
%% Filename: chap02.tex
%% Encoding: UTF-8
%% Author: Yuan Xiaoshuai - yxshuai@gmail.com
%% Created: 2012-04-27 19:37
%% Last modified: 2019-11-01 11:43
%%================================================
\chapter{模板简介及安装}
\label{cha:introduction}

\zzuthesis(\textbf{Z}heng\textbf{z}hou \textbf{U}niversity \textbf{Thesis})以
标准book类文档为基础,根据清华大学学位论文模板修改而来。本科毕业设计(论文)
和研究生学位论文(含硕士和博士)分别根据《郑州大学材料科学与工程学院本科毕业
设计(论文)基本规范》和《郑州大学学位论文写作规范格式》的规范要求定制,目前
该模板已基本符合相关《规范》的要求。除格式方面的修改外,还参考其它论文模板,
尽可能简化了代码,更有利于初学者使用。

\textsf{模板作者声明:}该模板非学校官方模板,尽管模板基本满足《规范》要求,并
参考2011年通过审核的学位论文进行调整,但难免还存在不足之处,欢迎大家反馈。

% \section{本科毕业设计论文内容要求}
% \label{sec:bachelor}
% 
% 根据《郑州大学材料科学与工程学院本科毕业设计(论文)基本规范》的要求,本科
% 毕业设计(论文)的装订如下:封面(格式版式固定,只替换文字),中文摘要,英
% 文摘要,目录,1. 前沿(概述),2. 实验过程(方法),3. 实验结果与讨论,4.
% 结论,5 . 参考文献,6. 表格附件(毕业设计(论文)任务书、郑州大学毕业论文开
% 题报告表、毕业设计(论文)计划进程表、郑州大学毕业设计中期检查表、毕业论文
% (论文)成绩评定表),7. 外文翻译,8. 外文原文,9. 致谢。
% 
% \section{研究生论文内容要求} \label{sec:graduates}
% 
% 根据二〇〇九年十月郑州大学学位办公室编制的《郑州大学博士、硕士学位论文写作
% 规范》的要求,学位论文一般由十五部分组成,依次为:1. 封面;2.英文封面;3.
% 学位论文原创性声明;4. 学位论文使用授权声明;5. 摘要;6. Abstract;7. 目录
% ;8.图和附表清单;\footnote{模板将图和附表清单分为插图清单和附表清单两个部
% 分}9.符号说明;10. 正文;11.注释;12. 参考文献;13. 致谢;14. 附录;15. 个
% 人简历、在学期间发表的学术论文及研究成果。
  
\section{模板系统需求}
\label{sec:requirements}

\TeXLive{}为多种基于Unix的平台提供了可执行文件\footnote{\TeXLive{}的安装参见:
\url{http://tug.org/texlive/acquire.html}。},包括GNU/Linux、Mac OS X、和
Cygwin,并支持Windows XP及后续版本。理论上该模板应该在常见的系统平台上都可以正
常运行,但仅在以下系统平台下进行了测试:
\begin{enumerate}
\item Fedora 16 + \TeXLive{} 2011 + Jabref 2.8
\item Windows XP + \TeXLive{} 2011 + Jabref 2.8
\end{enumerate}

由于模版文件中用到的部分命令在最新的发行版中有所更改,故在 2016 年 8 月 28 日
对模版进行修改,并在以下平台测试通过:

\begin{itemize}
\item CentOS 7.2.1511 + \TeXLive{} 2015
\end{itemize}

\section{模板的组成部分}

下表列出了 \zzuthesis{} 的主要文件及其功能介绍:
\begin{center}
\tablefirsthead{%
\hline
{\hei 文件(夹)} & {\hei 功能描述}\\\hline\hline}
\tablehead{\hline
{\hei 文件(夹)} & {\hei 功能描述}\\\hline\hline}
% \endhead
\tabletail{\hline}
% \endfoot
  \begin{mpsupertabular}{l|p{8cm}}
zzuthesis.cls & 模板类文件\\
zzuthesis.cfg & 模板配置文件\\
zzubib.bst & 参考文献样式文件\\\hline
docutils.sty & 模板示例文档用到的宏包及定义\footnote{使用该模板时可根据需要加
入其它自定义宏包及定义。}\\ 
main.tex & 示例文档主文件\\
spine.tex & 书脊示例文档\\
a3cover.tex & A3封面示例文档\\
ref/ & 示例文档参考文献目录\\
data/ & 示例文档章节具体内容\\
figures/ & 示例文档图片路径\\
fonts/ & 模版中用到的字体文件\\\hline
Makefile & Linux 自动编译工具\\
msmake.cmd & Windows 批处理工具\\
  \end{mpsupertabular}
\end{center}

该模板可以通过以下途径获取:
\begin{itemize}
\item 浏览器下载:
\href{https://codeload.github.com/tuxify/zzuthesis/zip/master}{Github Download Zip}
\item Git 命令:
\texttt{git clone https://github.com/tuxify/zzuthesis.git}
\end{itemize}

\section{准备工作}

模板中调用的宏包%有:
% \begin{center}
% \begin{minipage}{1.0\linewidth}\centering
% \begin{tabular}{*{7}{l}}\hline
% calc & titletoc & amsmath & amssymb & amsthm & \\% txfonts 
% geometry & tabularx & multirow & longtable & booktabs & subfig\footnote{版本要求:$\geq$2005/06/28 ver: 1.3} \\
% graphicx & indentfirst\footnote{2016/8/28修改后首段不缩进,添加该宏包后解决} & paralist & xeCJK & natbib & hyperref \\
% hypernat & & & & &\\\hline
% \end{tabular}
% \end{minipage}
% \end{center}
% 
% 这些包
在常见的 \TeX{} 系统中都有,如果没有请到 \url{www.ctan.org} 下载。

模板中用到的字体采用 ctex 文档中的字体配置,需要注意的是,研究生论文封面用到华文行楷字体(STXingkai),请自行安装该字体。

\section{示例文档的生成}

示例文档的生成需要多次运行\XeLaTeX{},直到不再出现警告为止,详细过程如下:
\begin{code}
1. 发现文档的引用关系,文件后缀 .tex 可以省略
$ xelatex main
2. 编译参考文件源文件,生成 bbl 文件
$ bibtex main
3. 解决引用
$ xelatex main
4. 生成完整的 pdf 文件
$ xelatex main
\end{code}

由于一般情况下的使用方法每次需要输入数次命令,比较麻烦,故模板提供了一些自动
处理的文件。

Linux平台可以使用 \emph{Makefile} 文件:
\begin{code}
$ make thesis    # 生成示例文档
$ make a3cover   # 生成A3封面
$ make clean     # 清除临时文件
\end{code}

对于Windows平台,则提供了一个批处理脚本 \emph{msmake.cmd}\/:
\begin{code}
your_path $ msmake thesis  # 生成示例文档
your_path $ msmake a3cover # 生成A3封面
your_path $ msmake clean   # 清除临时文件
\end{code}
